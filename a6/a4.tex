% LaTeX Article Template - customizing header and footer
\documentclass{article}

\newtheorem{thm}{Theorem}

% Set left margin - The default is 1 inch, so the following
% command sets a 1.25-inch left margin.
% \setlength{\oddsidemargin}{0.05in}

% Set width of the text - What is left will be the right margin.
% In this case, right margin is 8.5in - 1.25in - 6in = 1.25in.
\setlength{\textwidth}{8in}

% Set top margin - The default is 1 inch, so the following
% command sets a 0.75-inch top margin.
\setlength{\topmargin}{-0.05in}

% Set height of the header
\setlength{\headheight}{0.2in}

% Set vertical distance between the header and the text
\setlength{\headsep}{0.1in}

% Set height of the text
\setlength{\textheight}{8.5in}

% Set vertical distance between the text and the
% bottom of footer
\setlength{\footskip}{0.1in}

% Set the beginning of a LaTeX document
\usepackage{multirow}
\usepackage{fullpage}
\usepackage{graphicx}
\usepackage{amsthm}
\usepackage{amssymb}
\usepackage{amssymb}
\usepackage{algpseudocode}
\usepackage{listings}
\usepackage{color}
\usepackage{float}
\usepackage{enumitem}

\definecolor{dkgreen}{rgb}{0,0.6,0}
\definecolor{gray}{rgb}{0.5,0.5,0.5}
\definecolor{mauve}{rgb}{0.58,0,0.82}

\lstset{language=sql,
  aboveskip=2mm,
  belowskip=2mm,
  columns=flexible,
  basicstyle={\small\ttfamily},
  numbers=none,
  numberstyle=\tiny\color{gray},
  keywordstyle=\color{blue},
  commentstyle=\color{dkgreen},
  stringstyle=\color{dkgreen},
  breaklines=true,
  breakatwhitespace=true,
  tabsize=3
}

\graphicspath{%
    {converted_graphics/}% inserted by PCTeX
    {/}% inserted by PCTeX
}
%%%%%%%%%%%%%%%%%%%%%%%%%%%%%

\begin{document}\title{Assignment $6$\\ Computer Science \\ Fall 2017\\ B461}         % Enter your title between curly braces
\author{Steven Myers}        % Enter your name between curly braces
\date{\today}          % Enter your date or \today between curly braces
\maketitle

% Redefine "plain" pagestyle
\makeatother     % `@' is restored as a "non-letter" character

% Set to use the "plain" pagestyle
\pagestyle{plain}

\section*{Answers}

\begin{enumerate}
    \item % problem one
        \begin{enumerate}

        \item % (a)
        \begin{displaymath}
            \pi_{sid, bookno}
                (\sigma_{Buys.bookno \ne Buys2.bookno \wedge Student.name='Eric' \wedge Buys.bookno \ne '2010'}
                    (Student \times\ Buys \times\ Buys2))
        \end{displaymath}

        \item % (b)
        \begin{enumerate}

            \item Original Query
            \begin{lstlisting}
                SELECT DISTINCT b.bookno, b.title
                FROM book b, student s
                WHERE b.price = SOME(select b1.price
                                     from buys t, book b1
                                     where b1.price > 50 and
                                     s.sid = t.sid and
                                     t.bookno = b1.bookno);
            \end{lstlisting}

            \item Convert from using SOME to an exists statement
            \begin{lstlisting}
            SELECT DISTINCT b.bookno, b.title
            FROM book b, student s
            WHERE EXISTS(select 1
                         from buys t, book b1
                         where b1.price > 50 and
                         s.sid = t.sid and
                         t.bookno = b1.bookno);
            \end{lstlisting}

            \item Push down student $s$ relation into subquery as $s1$.
            \begin{lstlisting}
            SELECT DISTINCT b.bookno, b.title
            FROM book b, student s
            WHERE EXISTS(select 1
                         from buys t, book b1, student s1
                         where b1.price > 50 and
                         s1.sid = t.sid and
                         t.bookno = b1.bookno);
            \end{lstlisting}

            \item The inner query can be translated as:
            \begin{displaymath}
                \pi_{b1.price}(
                    \sigma_{b1.price>50 \wedge s1.sid=t.sid \wedge t.bookno=b1.bookno}(T \times B1 \times S1
                    ))
            \end{displaymath}

            \item Finally, we perform the semi-join with the inner query and the student and book relation:
            \begin{displaymath}
                \pi_{b.bookno, b.title}(S \times B) \ltimes
                \pi_{b1.price}(
                    \sigma_{b1.price>50 \wedge s1.sid=t.sid \wedge t.bookno=b1.bookno}(T \times B1 \times S1
                    ))
            \end{displaymath}



        \end{enumerate}

        \item % (c)

        \begin{enumerate}
            \item Original Query
            \begin{lstlisting}
            SELECT b.bookno
            FROM book b
            WHERE b.bookno IN
            (SELECT b1.bookno FROM book b1 WHERE b1.price > 50)
            UNION
            (SELECT c.bookno FROM cites c);
            \end{lstlisting}

            \item RA representation of inner query (simple translation)
            \begin{displaymath}
                \pi_{bookno}(\sigma_{price > 50}(book)) \cup \pi_{bookno}(cites)
            \end{displaymath}

            \item Since the IN predicate is equivalent to saying that there exists one bookno for which a bookno in the inner query matches, we can use the semi join

            \begin{displaymath}
                \pi_{bookno}(book) \ltimes \pi_{bookno}(\sigma_{price > 50}(book)) \cup \pi_{bookno}(cites)
            \end{displaymath}

        \end{enumerate}

        \item % (d)

        \begin{enumerate}
            \item Original Query
            \begin{lstlisting}
                SELECT b.bookno FROM book b
                WHERE b.price >= 80 and
                NOT EXISTS(SELECT b1.bookno
                            FROM book b1
                            WHERE b1.Price > b.Price);
            \end{lstlisting}

            \item First, push down book b relation into subquery as book b2.
            \begin{lstlisting}
                SELECT b.bookno FROM book b
                WHERE b.price >= 80 and
                NOT EXISTS(SELECT b1.bookno
                            FROM book b1, book b2
                            WHERE b1.Price > b2.Price);
            \end{lstlisting}

            \item Now, we can properly translate the inner query as:
            \begin{displaymath}
                \pi_{b1.bookno}(\sigma_{b1.price>b2.price}(B1 \times B2))
            \end{displaymath}

            \item To preserve the semantics of the original outer query, we need to perform an anti-semi join:
            \begin{displaymath}
                \pi_{b.bookno}(\sigma_{b.price >= 80}(B \ \overline{\ltimes}\ \pi_{b1.bookno}(\sigma_{b1.price>b2.price}(B1 \times B2))))
            \end{displaymath}

        \end{enumerate}
        \newpage
        \item % (e)
        \begin{enumerate}
            \item Original query:
            \begin{lstlisting}
            SELECT s.sid
            FROM Student s
            WHERE EXISTS(SELECT 1
                        FROM Book b
                        WHERE b.price > 50 AND
                        b.bookno IN (SELECT t.bookno
                                    FROM Buys t
                                    WHERE s.sid = t.sid AND
                                    s.sname = 'Eric'))

            \end{lstlisting}

            \item First, push down parameterized values in the inner-most query. We will push down the student $s$ relation as $s1$.
            \begin{lstlisting}
            SELECT s.sid
            FROM Student s
            WHERE EXISTS(SELECT 1
                        FROM Book b
                        WHERE b.price > 50 AND
                        b.bookno IN (SELECT t.bookno
                                    FROM Buys t, Student s1
                                    WHERE s1.sid = t.sid AND
                                    s1.sname = 'Eric'))

            \end{lstlisting}

            \item Next, convert the first inner subquery IN expression into an EXISTS statement. We can do this by pushing the book relation completely into the inner most subquery and adding its conditions
            \begin{lstlisting}
            SELECT s.sid
            FROM Student s
            WHERE EXISTS(SELECT t.bookno
                                    FROM Buys t, Student s1, Book b
                                    WHERE s1.sid = t.sid AND
                                    b.price > 50 AND
                                    b.bookno = t.bookno
                                    s1.sname = 'Eric')

            \end{lstlisting}

            \item We can translate the inner query as:
            \begin{displaymath}
                \pi_{t.bookno}(\sigma_{s1.sid=t.sid\ \wedge\ b.price > 50\ \wedge\ b.bookno = t.bookno\ \wedge\ s1.sname='Eric'}(S1 \times B \times T))
            \end{displaymath}

            \item Finally, we can translate our outer exists expression with a semijoin and project the student sid.
            \begin{displaymath}
                \pi_{s.sid}(S \ltimes
                \pi_{t.bookno}(\sigma_{s1.sid=t.sid\ \wedge\ b.price > 50\ \wedge\ b.bookno = t.bookno\ \wedge\ s1.sname='Eric'}(S1 \times B \times T)))
            \end{displaymath}

        \end{enumerate}

        \newpage

        \item % (f)
        \begin{enumerate}

            \item Original query:
            \begin{lstlisting}
            SELECT s1.sid, s2.sid
            FROM student s1, student s2
            WHERE s1.sid <> s2.sid AND
            NOT EXISTS(SELECT 1
                        FROM Buys t1
                        WHERE t1.sid = s1.sid AND
                        t1.bookno NOT IN (SELECT t2.bookno
                                        FROM Buys t2
                                        WHERE t2.sid = s2.sid));
            \end{lstlisting}

            \item For the deepest query, we need to translate the NOT IN statement to NOT EXISTS.
            \begin{lstlisting}
            SELECT s1.sid, s2.sid
            FROM student s1, student s2
            WHERE s1.sid <> s2.sid AND
            NOT EXISTS(SELECT 1
                        FROM Buys t1
                        WHERE t1.sid = s1.sid AND
                        NOT EXISTS (SELECT 1
                                    FROM Buys t2
                                    WHERE t2.sid = s2.sid AND t1.bookno <> t2.bookno));
            \end{lstlisting}

            \item We now need to push down our relations recursively from the top and middle queries to the deepest query.
            \begin{lstlisting}
            SELECT s1.sid, s2.sid
            FROM student s1, student s2
            WHERE s1.sid <> s2.sid AND
            NOT EXISTS(SELECT 1
                        FROM Buys t, Student s1, Student s2
                        WHERE t.sid = s1.sid AND
                        NOT EXISTS (SELECT t2.sid, t1.bookno, t2.bookno, s1.sid
                                    FROM Buys t2, Buys t1, Student S1, Student S2
                                    WHERE t2.sid = s2.sid AND t1.bookno <> t2.bookno));
            \end{lstlisting}

            \item Now that we've pushed down the upper query's relations into deeper queries, we can translate the deepest
            query as:
            \begin{displaymath}
                \varepsilon = \pi_{t2.sid, t1.bookno, t2.bookno, s1.sid}(\sigma_{t2.sid=s2.sid \wedge t1.bookno \ne t2.bookno}(T1 \times T2 \times S1 \times S2))
            \end{displaymath}

            \item Our middle query then can be translated as:
            \begin{displaymath}
                \tau = \pi_{s1.sid, s2.sid, t.bookno}(\sigma_{t.sid=s1.sid}(T \times S1 \times S2) \overline{\ltimes} \varepsilon)
            \end{displaymath}

            \item Finally, we can semijoin the top level query in an expression like so:
            \begin{displaymath}
                \pi_{s1.sid, s2.sid}(\sigma_{s1.sid \ne s2.sid}(S1 \times S2) \overline{\ltimes} \tau)
            \end{displaymath}


        \end{enumerate}

    \end{enumerate}

    \item  % problem two
    \begin{enumerate}

    \item % (a)
    \begin{enumerate}
        \item % (a)
        Original RA Expression:
        \begin{displaymath}
            \pi_{sid, bookno}
                (\sigma_{Buys.bookno \ne Buys2.bookno \wedge Student.name='Eric' \wedge Buys.bookno \ne '2010'}
                    (Student \times\ Buys \times\ Buys2))
            \end{displaymath}
    \end{enumerate}


    \item % (b)
    \begin{enumerate}
        \item Original RA Expression: % (b)
        \begin{displaymath}
            \pi_{b.bookno, b.title}(S \times B) \ltimes
            \pi_{b1.price}(
                \sigma_{b1.price>50 \wedge s1.sid=t.sid \wedge t.bookno=b1.bookno}(T \times B1 \times S1
                ))
        \end{displaymath}
    \end{enumerate}


    \item % (c)
    \begin{enumerate}
        \item Original RA Expression: % (c)
        \begin{displaymath}
            \pi_{b.bookno}(\sigma_{b.price >= 80}(B \ \overline{\ltimes}\ \pi_{b1.bookno}(\sigma_{b1.price>b2.price}(B1 \times B2))))
        \end{displaymath}
    \end{enumerate}

    \item % (d)
    \begin{enumerate}
        \item Original RA Expression % (d)
        \begin{displaymath}
            \pi_{b.bookno}(\sigma_{b.price >= 80}(B \ \overline{\ltimes}\ \pi_{b1.bookno}(\sigma_{b1.price>b2.price}(B1 \times B2))))
        \end{displaymath}
    \end{enumerate}

    \item % e
    \begin{enumerate}
        \item Original RA Expression: % (e)
        \begin{displaymath}
            \pi_{s.sid}(S \ltimes
            \pi_{t.bookno}(\sigma_{s1.sid=t.sid\ \wedge\ b.price > 50\ \wedge\ b.bookno = t.bookno\ \wedge\ s1.sname='Eric'}(S1 \times B \times T)))
        \end{displaymath}

    \end{enumerate}

    \item % f
    \begin{enumerate}
        \item Original RA Expression: % (f)
        \begin{displaymath}
            \varepsilon = \pi_{t2.sid, t1.bookno, t2.bookno, s1.sid}(\sigma_{t2.sid=s2.sid \wedge t1.bookno \ne t2.bookno}(T1 \times T2 \times S1 \times S2))
        \end{displaymath}

        \begin{displaymath}
            \tau = \pi_{s1.sid, s2.sid, t.bookno}(\sigma_{t.sid=s1.sid}(T \times S1 \times S2) \overline{\ltimes} \varepsilon)
        \end{displaymath}

        \begin{displaymath}
            \pi_{s1.sid, s2.sid}(\sigma_{s1.sid \ne s2.sid}(S1 \times S2) \overline{\ltimes} \tau)
        \end{displaymath}
    \end{enumerate}

\end{document}
