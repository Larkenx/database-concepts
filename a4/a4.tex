% LaTeX Article Template - customizing header and footer
\documentclass{article}

\newtheorem{thm}{Theorem}

% Set left margin - The default is 1 inch, so the following
% command sets a 1.25-inch left margin.
% \setlength{\oddsidemargin}{0.05in}

% Set width of the text - What is left will be the right margin.
% In this case, right margin is 8.5in - 1.25in - 6in = 1.25in.
\setlength{\textwidth}{8in}

% Set top margin - The default is 1 inch, so the following
% command sets a 0.75-inch top margin.
\setlength{\topmargin}{-0.25in}

% Set height of the header
\setlength{\headheight}{0.3in}

% Set vertical distance between the header and the text
\setlength{\headsep}{0.2in}

% Set height of the text
\setlength{\textheight}{8.5in}

% Set vertical distance between the text and the
% bottom of footer
\setlength{\footskip}{0.1in}

% Set the beginning of a LaTeX document
\usepackage{multirow}
\usepackage{fullpage}
\usepackage{graphicx}
\usepackage{amsthm}
\usepackage{amssymb}
\usepackage{amssymb}
\usepackage{algpseudocode}
\usepackage{listings}
\usepackage{color}
\usepackage{float}
\usepackage{enumitem}

\definecolor{dkgreen}{rgb}{0,0.6,0}
\definecolor{gray}{rgb}{0.5,0.5,0.5}
\definecolor{mauve}{rgb}{0.58,0,0.82}

\lstset{frame=tb,
  language=python,
  aboveskip=3mm,
  belowskip=3mm,
  showstringspaces=false,
  columns=flexible,
  basicstyle={\small\ttfamily},
  numbers=none,
  numberstyle=\tiny\color{gray},
  keywordstyle=\color{blue},
  commentstyle=\color{dkgreen},
  stringstyle=\color{mauve},
  breaklines=true,
  breakatwhitespace=true,
  tabsize=3
}

\graphicspath{%
    {converted_graphics/}% inserted by PCTeX
    {/}% inserted by PCTeX
}
%%%%%%%%%%%%%%%%%%%%%%%%%%%%%

\begin{document}\title{Assignment $4$\\ Computer Science \\ Fall 2017\\ B461}         % Enter your title between curly braces
\author{Steven Myers}        % Enter your name between curly braces
\date{\today}          % Enter your date or \today between curly braces
\maketitle

% Redefine "plain" pagestyle
\makeatother     % `@' is restored as a "non-letter" character

% Set to use the "plain" pagestyle
\pagestyle{plain}

\section*{Answers}

\begin{enumerate}
    \item % problem one
    Find the bookno of each book that is cited by at least one book that cost cost less than \$50.

    \begin{displaymath}
            \pi_{C.CitedBookno}(\sigma_{B.Price\ <\ 50}\ (Cites\bowtie_{C.CitedBookno=B.Bookno} Book))
    \end{displaymath}

    \item % problem two
    Find the bookno and title of each book that was bought by a student who majors in CS and in Math. Let Major1 be a copy of Major with schema ($sid,major$).

    \begin{displaymath}
        \pi_{B.Bookno, B.Title}( \sigma_{M.Major=Math\ \wedge\ M1.Major=CS}(Student \bowtie (Major \bowtie Major1))) \bowtie Buys \bowtie Book)
    \end{displaymath}
    % \pi_{B.Bookno, B.Title}(\sigma_{M.Major='CS'\ \vee\ M.Major='Math'}(Student \bowtie Major \bowtie Buys \bowtie Book))

    \item % problem three
    Find the sid-bookno pairs (s, b) pairs such student s bought book b and
    such that book b is cited by at least two books that cost less than \$50. Let Cites1 be a copy of Cites with schema (Bookno, CitedBookNo).

    \begin{displaymath}
        % wanna find books cited by two books
        \pi_{CitedBookno}(
        \sigma_{Price\ <\ 50} (\sigma_{C.CitedBookno=C1.CitedBookno\ \wedge\ C.BookNo \ne C1.BookNo}(Cites \times Cites1) \bowtie^1 Book)) \bowtie Buys
    \end{displaymath}

    \textit{$^1$ Should be\ $\bowtie_{CitedBookno=Book.Bookno}$ but can't fit in one line}

    \item % problem four
    Find the sid of each student who bought all books that cost more than \$50.
    \begin{displaymath}
        \pi_{sid}(Student) -
            \pi_{sid}((Student\ \times\ (\pi_{bookno}(\sigma_{Price > 50}(Book)))) % cartesian product of students/books over 50
            - Buys)] % minus those who didn't purchase all of the books
    \end{displaymath}

    \item % problem five
    Find the Bookno of each book that was not bought by any student who majors in CS.

    \begin{displaymath}
        (\pi_{bookno}(Book) - \pi_{bookno}(Buys)) \cup (\pi_{bookno}(Book) - \pi_{bookno}(\sigma_{Major=CS}(Buys \bowtie Major)))
    \end{displaymath}

    \item % problem six
    Find the Bookno of each book that was not bought by all students who majors in CS.
    \begin{displaymath}
        Let\ R=\pi_{bookno}(Book) - \pi_{bookno}(Buys)
    \end{displaymath}
    \begin{displaymath}
        % \pi_{bookno}(\pi_{sid}(\sigma_{Major=CS}(Major)) \bowtie Buys) % all the books purchased by CS students
        % books purchased by a CS student with at least one other CS student who didn't buy that book
        % -  \pi_{bookno}(\sigma_{T.Bookno \ne T1.Bookno\ \wedge\  }((Buys \bowtie Major) \times (Buys1 \bowtie Major)) )
        % cross every possible book with every CS student, subtract the actual CS student buys. Any book remaining is a book not
        % purchased by all students
        \pi_{sid, bookno}(\pi_{sid}(\sigma_{Major=CS}(Major)) \times \pi_{bookno}(Book)) - \pi_{sid, bookno}(\sigma_{Major=CS}(Buys \bowtie Major)) \cup R
    \end{displaymath}

    *\textit{I created a cartesian product of CS students and all possible books then did the set difference of that cartesian product
    with all the CS student's purchased books. Any book remaining from that list must not have been purchased by every student. I assigned R1 to fit everything on single lines.}

    \item % problem seven
    Find sid-bookno pairs (s, b) such that not all books bought by student s are books that cite book b.

    \item % problem eight
    Find sid-bookno pairs (s,b) such student s only bought books that cite book b.

\end{enumerate}
\end{document}
